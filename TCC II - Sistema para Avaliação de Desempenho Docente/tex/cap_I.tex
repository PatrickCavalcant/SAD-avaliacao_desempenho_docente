\chapter{Introdução} \label{introdução}

    A partir de 1990, começa a ser implantada no Brasil a avaliação de desempenho, apoiando nos pilares da burocracia, mas revestida de novos ideais, dentre: o foco na desburocratização, descentralização, a autonomia do gestor, a horizontalização e a maior participação da comunidade \cite{bresser}. Desse contexto, percebe-se que a racionalidade existente nas políticas educacionais do país tem a considerar o docente como agente essencial para o avanço da qualidade de ensino e demandar que o trabalho e a capacitação desses profissionais sejam avaliados. Surgindo a necessidade para elaboração de planos de carreira que potencializa a adoção da avaliação de desempenho docente.

    A Lei de Diretrizes e Bases da Educação (LDB), no art. 67, estabelece que, os sistemas de ensino promoverão a valorização dos profissionais da educação, assegurando-lhes, inclusive nos termos dos estatutos e dos planos de carreira do magistério \cite{lei9394}. A existência da avaliação de desempenho indica que: progressão funcional baseada na titulação ou habilitação, e na avaliação do desempenho \cite{lei9394}, são determinações que possuem a garantia para alcançar a progressão na carreira profissional, sendo submetido a avaliação e possuindo a titulação e habilitação necessárias.

    Considerando que a aprovação em avaliação de desempenho individual consta como requisito legal a ser observado e atendido para aquisição de direito à progressão funcional e promoção na carreira docente, conforme disciplinado pela Portaria MEC nº 554, de 20 de junho de 2013 \cite{reitoria}. A Comissão Permanente de Pessoal Docente (CPPD) do Instituto Federal de Goiás (IFG) (CPPD, Portaria nº 1792/2020), a fim de dar prosseguimento à rotina para promoção e progressão dos docentes do quadro permanente de pessoal do IFG efetuou a criação da Avaliação Docente. Com isso, neste trabalho descreve-se as diversas etapas do desenvolvimento de um Sistema de Avaliação Docente (SAD) para o Instituto Federal de Goiás.

    Foi verificado a necessidade da instituição em um sistema de avaliação de desempenho docente, onde observado que o modelo de avaliação aplicado pelos sistemas atuais, já não agrega resultados satisfatórios diante a necessidade atual do IFG. Sendo que o mesmo possui uma grande quantidade de dados que são mantidos e unificados manualmente. A utilização de um novo \textit{software} assumiria uma integridade no controle desses dados gerados. Ao desenvolver um SAD, é construído um sistema que contemple as recomendações da instituição, desde as ferramentas utilizadas, passando pela escolha das tecnologias de desenvolvimento, a preparação do sistema e a adequada da utilização dos resultados, até os métodos de implantação a ações de manutenção.

    Serão também abordadas as metodologias ágeis aplicadas na criação e entrega do produto, de forma que atendam o conjunto de práticas eficazes, que se destinam a produzir o sistema demonstrando uma qualidade em seus processos, em uma forma de gerenciar o projeto sendo adaptável às mudanças. Essa abordagem propôs melhorias na gestão e acelerar os desenvolvimento do projeto.

    Com base nisso, este projeto tem como objetivo apresentar a necessidade de um Sistema de Avaliação Docente (SAD) para o apoio das demandas de avaliação de desempenho docente da Comissão Permanente de Pessoal Docente em um sistema unificado e entregar uma melhoria importante para a organização, podendo assim facilitar o processo de progressão dos docentes da instituição.


\section{Problema e Hipótese}

    O problema escolhido foi: como auxiliar a equipe da CPPD na otimização do processo de avaliação de desempenho docente do Instituto Federal de Goiás?
    
    A inexistência de um sistema estruturado para avaliação de desempenho, somado à alta demanda, resulta em um grande trabalho manual dos professores responsáveis, pertencentes a CPPD, o projeto apoiará o processo de avaliação dos docentes, coordenadores de curso, coordenadores acadêmicos e/ou chefe de departamento.
    
    Nesse cenário, este trabalho visa atender a demanda da CPPD, e também mostrar que é possível desenvolver um sistema, que seja capaz de gerir as informações e organizar o processo. O sistema será responsável pela condução do processo de avaliação docente para a efetivação da progressão funcional dos mesmos.
    
    Surge então a necessidade da criação do \textit{software}, onde as avaliações serão armazenadas e servirão como histórico para esses colaboradores, e também serão utilizadas para melhoria e estruturação deste processo, tirando esses registros dos formulários \textit{on-line} que atualmente consistem em um sistema externo, não vinculado a instituição que contempla parte das avaliações e do sistema Q-Acadêmico responsável pela outra parcela das avaliações de desempenho docente. Neste trabalho será apresentado como o Instituto Federal de Goiás pode agregar com uma solução de pesquisa, que tem como proposta construir um Sistema de Avaliação Docente (SAD). Também serão observados seus benefícios como a redução de custos e o aumento da agilidade, da otimização em processos, da acessibilidade e disponibilidade e da segurança junto à CPPD.
    



\section{Justificativa}

    De acordo \citen{comissao} as Instituições de Ensino Superior (IES), de um modo geral, vem sendo alvo de inúmeras questões sobre sua atuação no contexto social, e a ausência de subsídios que apresentem respostas concretas às questões constantes tem provocado o descrédito quanto à responsabilidade social. Desta forma, surge no seu bojo uma latente questão: As Instituições de Ensino Superior vem atendendo à demanda e expectativas da sociedade brasileira, enquanto entidade responsável pela disseminação do conhecimento? 
    
    Conforme abordado por \citen{oliveira-castro} os sistemas de avaliação devem ser justos e imparciais, baseados em padrões de desempenho atingíveis, objetivos, claros e apoiados na realidade dos cargos ou postos de trabalho. Para tal, é necessário pesquisar os padrões desejáveis de desempenho junto aos ocupantes dos cargos e às respectivas chefias.
    
    Implantar um sistema de avaliação visa melhorar alguns pilares importantes para uma organização, podendo colocar como os principais: planejamento, produtividade, ética, desenvolvimento de carreira, excelência, melhoria no processo de trabalho, responsabilidade e comprometimento. Em geral, o sistema visa subsidiar informações a respeito do desempenho dos professores da Instituição, de modo a promover a melhoria do ensino. 

    Portanto, em suma, é importante destacar que as instituições estão cientes de seu valor no processo de desenvolvimento e crescimento institucional, considerando que os profissionais estão sendo exigidos cada vez mais se capacitar para o mercado. Logo a avaliação de desempenho apresenta-se como instrumento e ação capaz de sinalizar o desempenho e detectar alterações entre o planejado e o que está sendo executado, oferecendo, desta forma, subsídios para correção.

    Diante a esse cenário, o IFG apresenta aplicações ineficientes para a automatização dos processos de avaliações docentes. A CPPD possui uma necessidade de um \textit{software} que seja capaz de efetuar as avaliações dos docentes pertencentes a instituição, sendo apto de gerir as informações e consiga atender os requisitos conforme os parágrafos anteriores. O sistema visará entregar um processo integrado que seja preparado para criar e gerir as avaliações de desempenho para fins de progressão e promoção funcional.


\section{Objetivo} \label{Objetivo}
    Esta seção tem como finalidade apresentar os objetivos que precisam ser alcançados  para realização deste  projeto.

\subsection{Objetivo Geral} \label{Objetivo Geral}
    O objetivo geral deste projeto é melhorar o processo de avaliação de desempenho docente, contribuindo com a melhoria no processo de avaliação docente e posterior progressão funcional. Essa abordagem de avaliação irá auxiliar a descobrir os pontos fortes e fracos dos docentes, além do fato de que é uma forma importante para que os docentes entendam o processo de avaliação no trabalho. 


\subsection{Objetivos Específicos} \label{Objetivos Específicos}
    Com a finalidade de atingir o objetivo geral, este projeto tem os seguintes objetivos específicos:

    
\begin{itemize}
    \item {Fazer o levantamento de requisitos;}
    \item {Modelar os requisitos na forma de diagramas;}
    \item {Elaborar o projeto de software;}
    \item {Implementar o software;}
    \item {Apresentar o protótipo a CPPD;}
    \item {Efetuar os testes;}
    \item {Validar o protótipo;}
    \item {Avaliação do \textit{software} junto a CPPD;}

\end{itemize}




