\chapter{Introdução} \label{introdução}
    Neste trabalho descreve-se as diversas etapas do desenvolvimento de um Sistema de Avaliação Docente (SAD) para o Instituto Federal de Goiás (IFG) e discute-se sua aplicabilidade a outras organizações. Observamos a necessidade da instituição em um sistema de avaliação docente, onde foi levantado que o modelo aplicado já não agregaria resultado diante o cenário atual. Sendo que atualmente já possui uma grande quantidade de dados e com o apoio desse novo \textit{software} assumiria uma integridade no controle desses dados gerados.
       
    Ao desenvolver um SAD, conforme relatado neste trabalho procura-se construir um sistema que contemple as recomendações da instituição, desde as ferramentas já utilizadas, passando pela escolha das tecnologias de desenvolvimento, a preparação do sistema e a adequada da utilização dos resultados, até os métodos de implantação a ações de manutenção.
       
    Serão também abordadas as metodologias ágeis aplicadas na criação e entrega do produto, de forma que atendam o conjunto de práticas eficazes, que se destinam a produzir o sistema demostrando uma qualidade em seus processos, em uma forma de gerenciar o projeto sendo ele adaptável às mudanças.
       
    Com base nisso, este projeto de pesquisa tem como objetivo apresentar a necessidade de um SAD para o apoio das demandas de avaliação de desempenho docente da Comissão Permanente de Pessoal Docente (CPPD) e entregar uma melhoria importante para a organização, podendo assim facilitar o processo de progressão dos docentes da instituição. 
% ----------------------------------------------------------
       
%       A fim 
       
%       Diante uma demonstração de uma ’...’ crescente da organização. 
%       Além disso, o uso adequado da gestão e da informação, utilizando a retenção do conhecimento organizacional, pode ser revertido em vantagens competitivas para aplicações estratégicas.
       
%       Possíveis pontos de melhorias
%       A seguir são apresentados os modelos
       
%       Será abordado os seguintes pontos 
       

%    É preciso evitar o risco de construir instrumentos baseados unicamente em traços de personalidade (iniciativa, urbanidade etc.), mais suscetíveis a erros de avaliação do que fatores relativos à produtividade ou à qualidade do trabalho. Os fatores de avaliação devem ser claramente definidos e os instrumentos exemplificarem ações observáveis, de maneira que possam servir como indicadores de desempenho e referenciais seguros para atribuição de escores.
%    IMPLANTAÇÃO DE UM SISTEMA DE AVALIAÇÃO DE DESEMPENHO: MÉTODOS E ESTRATÉGIAS
    

% ----------------------------------------------------------
      
      
\section{Problema}
    A inexistência de um sistema estruturado para avaliação de desempenhos, somado a alta demanda, resulta em um grande trabalho manual dos professores responsáveis, pertencentes a CPPD, o projeto apoiará o processo de avaliação dos docentes, coordenadores de curso, coordenadores acadêmicos ou chefe de departamento.
    
    Nesse cenário, este trabalho visa atender a demanda da CPPD, e também mostra que é possível desenvolver um SAD, que seja capaz de gerir as informações e organizar o processo. O sistema será responsável pela condução do processo de avaliação docente para a efetivação da progressão funcional dos mesmos.
    
    À vista disso, surge a necessidade da criação do \textit{software}, onde estas avaliações serão armazenadas e servirão como histórico para esses colaboradores, e também servirão como melhoria e estruturação deste processo, tirando esses registros dos formulários \textit{on-line} que atualmente consistem em um sistema externo, não vinculado a instituição.
    
    Como auxiliar a equipe da CPPD na otimização de visualização e processamento dos dados da avaliação docente do Instituto Federal de Goiás?


\section{Justificativa}
    De acordo \citen{comissao} as Instituições de Ensino Superior, de um modo geral, vem sendo alvo de inúmeras questões sobre sua atuação no contexto social, e a ausência de subsídios que apresentem respostas concretas às questões constantes tem provocado o descrédito quanto à responsabilidade social. Desta forma, surge no seu bojo uma latente questão: As Instituições de Ensino Superior vem atendendo à demanda e expectativas da sociedade brasileira, enquanto entidade responsável pela disseminação do conhecimento? 
    
    Conforme abordado por \citen{oliveira-castro} os sistemas de avaliação devem ser justos e imparciais, baseados em padrões de desempenho atingíveis, objetivos e claros, apoiados na realidade dos cargos ou postos de trabalho. Para tal, é necessário pesquisar os padrões desejáveis de desempenho junto aos ocupantes dos cargos e às respectivas chefias.
    
    Implantar um sistema de avaliação visa melhorar alguns pilares importantes para uma organização, podendo colocar como os principais: planejamento, produtividade, ética, desenvolvimento de carreira, excelência, melhoria no processo de trabalho, responsabilidade e comprometimento.
    
    Portanto, em suma, é importante destacar que as instituições estão cientes de seu valor no processo de desenvolvimento e crescimento institucional, considerando que os profissionais estão sendo exigidos cada vez mais se capacitar para o mercado. Logo a avaliação de desempenho apresenta-se como instrumento e ação capaz de sinalizar o desempenho e detectar alterações entre o planejado e o que está sendo executado, oferecendo, desta forma, subsídios para correção.

    Diante a este cenário, o IFG não apresenta nenhuma aplicação para a automatização dos processos de avaliação docentes. A CPPD possui uma necessidade de um \textit{software} que seja capaz de efetuar as avaliações dos docentes pertencentes a instituição, sendo ele apto de gerir as informações e consiga atender os requisitos conforme os parágrafos anteriores. O sistema visará entregar um processo integrado que seja preparado para criar e gerir as avaliações de desempenho para fins de progressão e promoção funcional.

\section{Objetivo} \label{Objetivos}
    Esta seção tem como finalidade apresentar os objetivos que precisam ser alcançados para este projeto.

\subsection{Objetivos Gerais} \label{Objetivos Gerais}
    O objetivo deste \textit{software} é melhorar o processo de avaliação de desempenho docente, contribuindo com a melhoria no processo de avaliação docente e posterior progressão funcional. Essa abordagem irá ajudar a descobrir os pontos fortes e fracos, além do fato de que é uma forma importante para que os docentes entendam o processo de avaliação no trabalho. O sistema tem como intuito fornecer informações para CPPD, subsidiando assim as ações da comissão. 


\subsection{Objetivos Específicos} \label{Objetivos Específicos}
    Com a finalidade de atingir o objetivo geral, este projeto tem os seguintes objetivos específicos:

    
\begin{itemize}
    \item {Fazer o levantamento de requisitos;}
    \item {Modelar os requisitos na forma de diagramas;}
    \item {Elaborar o projeto de software;}
    \item {Implementar o software;}
    \item {Apresentar o protótipo a CPPD;}
    \item {Validar o protótipo;}
    \item {Efetuar os testes;}
    \item {Avaliação do \textit{software} junto a CPPD;}

\end{itemize}




